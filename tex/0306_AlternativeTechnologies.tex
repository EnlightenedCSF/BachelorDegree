\subsection{Анализ альтернативных технологий для навигации внутри помещений}

\begin{itemize}
    \item
    \textit{RFID} – способ автоматичесткой идендификации объектов, в котором посредством радиосигналов считываются или записываются данные, хранящиеся в так называемых транспондерах, или RFID-метках. По своей функциональности, RFID, как методы сбора информации, очень близки к штрих-кодам, наиболее широко применяемым сегодня для маркировки товаров. Следовательно, RFID хорошо подойдет, например, для отслеживания посылки по большому транспортному узлу, стоит лишь снабдить весь маршрут этой посылки и ее саму RFID-метками. Другими словами, RFID применяется в сценариях, когда существующая система имеет некоторое глобальное состояние. Маячки, напротив, предполагают лишь определении локального состояния.
    \item
    \textit{NFC} - технология беспроводной высокочастотной связи малого радиуса действия. Она предоставляет возможность обмена данными между устройствами, находящимися на расстоянии около 10 см. Вместе с существующей инфраструктурой бесконтактных карт, технология нацелена на использование в платежных системах и общественном транспорте. Надо заметить, что в своей сфере она обладает некоторыми важными преимуществами: NFC-метки потребляют мало энергии и легко встраиваются. \\ 
    Следует сделать замечание, что сам принцип работы технологий RFID и NFC не совсем подходит для решения круга задач, выдвинутых в рамках настоящей работы. На практике или их сложно встроить, ведь их требуется много (RFID), или радиус действия слишком мал, как в случае NFC.
    \item
    \textit{Навигация по Wi-Fi}. Используется уже существующая инфраструктура сетей связи – точки беспроводных сетей Wi-Fi, и это наименее затратный вариант. Методика определение координат следующая: устройство пользователя сканирует доступные Wi-Fi-точки доступа, затем информацию о них отправляет на сервер, где эти данные по базе данных сопоставляются с координатами этих точек доступа, по которым и вычисляются координаты пользователя. К сожалению, координаты Wi-Fi точек точно не известны, плюс могут меняться (перенесли Wi-Fi точку в другое место или заменили её на другую – координаты уже оказываются неверными). \\
    Точность при таком подходе оставляет желать лучшего (погрешность - до 25 метров! При использовании специально созданной Wi-Fi инфраструктуры точность достигает 3-5 метров, но это уже требует ощутимых затрат на создание и обслуживание подобной системы), да и идентицифировать клиентов по Wi-Fi, привязывая их расположение к карте помещений, проблематично: начиная с iOS 8, mac-адреса Apple-устройств (iPhone, iPad) постоянно меняются для предотвращения «рекламной» слежки.
    \item
    \textit{Геомагнитное позиционирование}. Основано на ориентировании по магнитному полю Земли и базируется на геомагнитных аномалиях как критериях для геомагнитного позиционирования (аномалии возникают вследствии неоднородности геомагнитного поля). Заключается в фиксации геомагнитных аномалий и нанесении их на карту территории, на которой предполагается ориентироваться. В дальнейшем навигация производится по составленной карте устройством, в которое встроен магнитометр. Практический пример реализации – система IndoorAtlas команды учёных из финского университета Оулу. \\
    Недостаток – высокая сложность реализации, невысокая точность. В помещениях очень много динамически меняющихся магнитных аномалий (проводка, поле в которой меняется в зависимости от подключённой нагрузки и сильно меняет конфигурацию магнитного поля вокруг себя; посетители со своими радиоэлектронными устройствами; стеллажи; тележки), сильно усложняющих навигацию, основанную на указанном способе ориентировании в пространстве.
    \item
    \textit{Системы спутниковой навигации (GPS/Глонасс и другие) + инерциальные навигационные системы (ИНС)}. Применимо, когда периодически появляется сигнал систем спутниковой навигации. Его точности по мнению современных исследователей достаточна, чтобы вычислить положение пользователя вплоть до десятков сантиметров \cite{web:GPScent}. Однако когда пользователь въезжает в тоннель, ещё доступны актуальные координаты и направление движения с GPS/Глонасс-спут\-ни\-ков, далее в самом тоннеле сигнал теряется, и поэтому используется уже инерциальная навигационная система (ИНС, основанная на базе акселерометра, гироскопа, магнитометра), которая использует в качестве начальных условий последние актуальные данные с GPS/Глонасс до потери связи со спутником и поддерживает их актуальность на основе получаемых с датчиков данных о текущей скорости, ускорении и направлении движения до возобновления связи со спутниками. \\
    Стоит принимать во внимание, что в ИНС ошибки постоянно накапливаются, и со временем данные, полученные с ИНС, становятся все более и более отличными от действительности.
    \item
    \textit{Ориентирование по базовым станциям операторов сотовой связи \\(GSM)}. В зоне видимости сотового телефона/GSM-модема постоянно находятся как минимум одна базовая станция (далее БС) GSM, а обычно их несколько. Координаты расположения этих БС известны (благодаря многочисленным навигационным сервисам (например <<Яндекс.На\-ви\-га\-тор>>), приложение получает информацию о видимых мобильным телефоном БС и текущем положении по GSM/<<Глонасс>>, и отправляет эти сведения в <<Яндекс>>, где на основе этих данных строится база соответствий <<БС-ко\-ор\-ди\-на\-ты>>, к которой имеется свободный доступ через предоставляемое API). Далее в модем отправляется команда \quotes{$AT+CREG=2$}, в результате чего можно получить сообщения +CREG: с информацией о текущей подключенной базовой станции: LAC и CELLID (соответственно код зоны и идентификатор БС). Отправив эти данные на один из специальных сервисов (предоставляемый <<Яндекс>>, "Google" и другими компаниями), возможно определить координаты этой БС. Многие модемы позволяют получить список видимых БС с указанием их LAC и CELLID - остаётся только через базы данных с координатами БС получить их координаты и методом триангуляции определить примерное местоположение пользователя. \\
    Минусы: невысокая точность (БС может быть удалена на расстоянии в 35км от пользователя, некоторые БС являются мобильными и постоянно меняют свою дислокацию).
    \item
    \textit{Использование Bluetooth-маячков iBeacon} даёт достаточную точность при приемлемом уровне финансовых затрат; перспективная технология, которая активно развивается, поэтому именно iBeacon был выбран в качестве предмета данной работы.
    \item
    \textit{Навигация, основанная на синергетическом эффекте} решает задачу определения текущего местоположения, используя все (или большинство) из перечисленных выше способов. Эффективность достигается за счёт того, что используется сразу несколько векторов определения координат, что способствует компенсации ошибок и повышению точности определения координат. На реализацию подобной системы в 2014 году фондом развития центра разработки и коммерциализации новых технологий <<Сколково>> был выделен грант в 1 млн долларов \cite{web:HabrBig}.    
\end{itemize}