\subsection{Анализ безопасности}

Не стоит переоценивать данную технологию и забывать о таком важном аспекте, как безопасность. На текущий момент развития трафик в процессе передачи никаким образом не шифруется, поэтому злоумышленник потенциально может провести атаку.

Выделяются 2 главных типов атак: спуффинг и пиггибэкинг \cite{web:SecuritySlides}.

\begin{itemize}
    \item
    \textit{Beacon Spoofing}. Атака основана на включении маячка или устройства, работающего как маячок, с параметрами, идентичными параметрам некоторой группы маячков. Это может быть использовано для того, чтобы симулировать некоторое событие или оповещение в месте, отличном от предполагаемого в контексте приложения. \\
    \textit{Пример угрозы:} предположим, есть некоторый магазин, использующий маячки для приветствия новых посетителей. Злоумышленник копирует настройки маячков, и впоследствие это позволяет ему разослать приветственное уведомление пользователям, которые находятся в абсолютно другой локации. Это может спровоцировать недоумение и, в конечном итоге, неудовлетворенность программным продуктом.
    \item
    \textit{Beacon Piggybacking, или Hijacking}. Атака основана на применении уже существующих маячков, предназначенных для некоторого приложения, в своем, стороннем приложении. Это может быть использовано для получении аналитики, основанной на оригинальный маячках, а также в рассылке сообщений и симуляции событий, не относящихся к предполагаемым. \\
    \textit{Пример угрозы:} пусть кофейня «А» конкурирует с кофейней «Б». «А» начинает использовать приложение с использованием маячков. В ответ кофейня «Б» разрабатывает свое приложение с информацией об идентификаторах чужих маячков. В результате каждый раз, как пользователь с установленным приложением от кофейни «Б» заходит в кофейню «А», на его устройство приходит оповещение о скидках на кофе в «Б».
\end{itemize}

Неприятный инцидент произошел на выставке потребительской электронике (Consumer Electronics Show – CES) в 2014 году. Для всех желающих была организована «охота за сокровищами», в рамках которой через приложение пользователи должны были определить место, где якобы зарыт сундук с кладом. Однако еще до начала мероприятия неизвестная группа хакеров взяла apk-файл приложения, проанализировала его структуру через декомпилятор, и смогла извлечь параметры всех используемых маячков. Это давало всю необходимую информацию, и победить таким способом можно было даже не выходя из дома \cite{web:ScavengeHunt}.

Выделяют 4 способа защиты:

\begin{enumerate}
    \item
    \textbf{Геолокационная проверка}. После получения оповещения от одного из новых (в рамках текущей сессии) маячков, устройство использует геолокационный сервис, чтобы убедиться, что маячок физически действительно находится поблизости.
    \item
    \textbf{Идентификация на основе начального значения (seed)}. Используются маячки с периодически меняющимся UUID. Алгоритм смены, в свою очередь, основан на некотором цифровом значении, хранимом отдельно. Через единый SDK происходит обновление, идентификация и синхронизация всего процесса. Маячок, не прошедший проверку – потенциальный вредитель – будет выкинут из рассмотрения.
    \item
    \textbf{Облачное подтверждение}. Способ базируется на основе предыдущего, но роль связующего звена на себя берет облачный сервис, а не локальный SDK. 
Этот механизм уже используется в маячках компании Estimote, и известен под названием “UUID Rotation”. Секретный ключ, определяющий смену идентификаторов, хранится на платформе Estimote Cloud.
    \item
    \textbf{Управление на уровне аппаратных средств}. Начальные параметры и их возможное обновление берут на себя аппаратные средства – контроллеры. При этом оповещение об обновлении приходит от облачного сервиса отдельно на контроллеры и отдельно на устройство пользователя. После этого контроллеры обновляют UUID маяков. В дальнейшем проверка маячков будет происходить на устройстве без использования сервиса.
\end{enumerate}

Преимущества и недостатки перечисленных методов представлены в таблицах 1 и 2, соответственно.

\begin{sideways}
	\begin{tabu} to \textheight { | X[l] | X[l] | X[l] | X[l] | X[l] | }
	\hline
	\textit{Способ защиты} & 
	    \textbf{Геолокационная проверка} & 
	    \textbf{Идентификация на основе генерации случайных чисел} &
	    \textbf{Облачное подтверждение} &
	    \textbf{Управление на уровне аппаратных средств} \\
	\hline
	
	\textit{Преимущества} & 
	    1. Наиболее дешевый способ \newline 
	    2. Простой в конфигурации и поддержке \newline
	    3. Может быть легко включен или выключен в любой момент \newline
	    4. Позволяет использовать традиционный iBeacon-формат для лучшей совместимости & 
	    
	    1. Позволяет защититься от перечисленных типов атак \newline
	    2. Нет привязки к геолокации или Интернет-соединению \newline
	    3. Не требует дополнительного оборудования &
	    
	    1. Позволяет в большей степени защититься от перечисленнных способов атак \newline
	    2. Не требуется подключение дополнительных устройств \newline
	    3. Усложняет процедуру проведения атаки для злоумышленника &
	    
	    1. Надежный уровень защиты \newline
	    2. Предоставляет устройство для обновления и обслуживания маячков \newline
	    3. Изменения легко применить к существующей системе \newline
	    4. Позволяет использовать традиционный iBeacon-формат для лучшей совместимости \\
	\hline
	\end{tabu}
\end{sideways}

\begin{sideways}
\begin{tabu} to \textheight { | X[l] | X[l] | X[l] | X[l] | X[l] | }
\hline
\textit{Способ защиты} & 
    \textbf{Геолокационная проверка} & 
    \textbf{Идентификация на основе генерации случайных чисел} &
    \textbf{Облачное подтверждение} &
    \textbf{Управление на уровне аппаратных средств} \\
\hline

\textit{Недостатки} & 
    1. Не защищает против piggybacking-атак \newline
    2. Геолокационная относительность не дает такой же точности, как остальные методы \newline
    3. Определение локации может являться причиной задержки в начале сессии &
    
    1. Сложен в модификации в случае неполадок \newline
    2. UUID маячков могут быть легко определены злоумышленником \newline
    3. Более сложное развертывание \newline
    4. Формат использования не поддерживается в приложениях Apple &
    
    1. Сложен в модификации в случае неполадок \newline
    2. Сложен в развертывании \newline
    3. Задержки в использовании могут отразиться на удобстве использовании приложения \newline
    4. Предполагает наличие Интернет-соединения \newline
    5. Формат использования не поддерживается в приложениях Apple &
    
    1. Наиболее дорогой тип размещения \newline
    2. Требует наличия дополнительного набора устройств \newline
    3. Не работает для маячков, расположенных на удалении от остальных \newline
    4. Предполагает наличие Интернет-соединения (периодически)\\
 \hline
\end{tabu}
\end{sideways}

\clearpage
\newpage

Примечательно, что Apple iOS SDK на уровне реализации не позволяет приложению сканировать эфир на обнаружение BLE-пакетов. Обязательным условием является явная конфигурация UUID, major и minor-идентификаторов \cite{web:CoreBlRestrict}.