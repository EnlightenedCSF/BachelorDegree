\section{Анализ задачи}

В случае спутниковой навигации (GPS/Глонасс) существуют сервисы Out\-Door, благодаря которым пользователь может узнать о ближайших кафе, ресторанах, гостиницах и других местах благодаря тому, что известно его текущее местоположение. А благодаря сервисам indoor-навигации возможно без проблем и оперативно найти ближайшую стойку регистрации в здании аэропорта, экспонат в музее, отдел и полку с нужным вам товаром в магазине, свободное место на парковке, и многое другое. 

В качестве наиболее значимых примеров можно перечислить следующие:
\begin{itemize}
    \item
    \textit{Розничная торговля.} Установив маячки, становится возможным приветствовать клиента, сообщать ему о новых поступлениях и предложениях, когда тот находиться рядом, а заодно узнать, как он движется внутри заведения и у каких товаров проводит больше времени, - верный ключ к оптимизации для маркетологов.
    \item
    \textit{Городской туризм.} Оказавшись рядом с объектом достопримечательности, пользователь получает информацию о месте. Этот же механизм поможет туристам получать уведомления о прибытии интересующего их транспорта к указанному месту.
    \item
    \textit{Управление кадрами.} Используя механизм маячков, становится легко следить за тем, сколько времени тратит в офисе каждый из сотрудников, если установить пару датчиков на вход. Также возможно быстро организовывать собрания: достаточно разослать оповещение каждому, кто не находится на рабочем месте.
    \item
    \textit{Фитнес-услуги.} Новички могут получать информацию о предназначении того или иного тренажера, а также инструкцию по его правильному использованию. Кроме того, для любого посетителя может оказаться полезной индивидуальная программа тренировок, которая бы не только замеряла время на подход и перерывы в занятиях, но и указала бы, где найти тренажер, наиболее эффективный для следующего этапа тренировок.
    \item
    \textit{Ресторанный бизнес.} В данной сфере оригинальной идеей будет выделение уникальных посетителей заведения: это не только информирует владельца ресторана, но и позволит разработать систему персональных скидок и условий. К примеру, клиент-завсегдатай может сформировать свой заказ заранее, и, как только он войдет в заведение, заказ автоматически поступит на кухню.
    \item
    \textit{Дом.} Bluetooth-маячки способны расширить функциональность умного дома или заменить многие сенсоры. В способы применения входят, например, умное освещение или блокирование входной двери. В данной сфере развивается, например, команда "airfy Beacon" \cite{web:airfyBeacon}. 
    \item
    \textit{Гостиницы.} Маячок на ресепшне улавливает гостей и сразу позволяет отобразить информацию о постоянных посетителях с их историей пребывания, что в итоге уменьшает время заселения. Такой системой пользуется сервис Mahana \cite{web:crunchTime}.
    \item
    \textit{Концертные площадки и стадионы.} Поклонники музыкантов и спортсменов получают подробную справочную информацию и статистику. Желающие могут приобрести билеты на будущие события. С нового сезона так работает приложение Главной лиги бейсбола США, на основных стадионах которой поставили маячки \cite{web:MLB}. 
    \item
    \textit{Дейтинг.} Маячки в барах улавливают людей с поддерживаемыми приложениями. Если посетители хотят с кем-нибудь познакомиться, то получат уведомление, через которое смогут увидеть профили тех, кто находится поблизости. Функцией уже вовсю пользуются конкуренты Tinder — например, Mingleton \cite{web:Mingleton}.
    \item
    \textit{Школы и университеты.} Преподавателям не нужно отмечать, кто из школьников и студентов ходит на занятия: система сама учитывает всех пришедших и, если необходимо, оповещает родителей. Кроме того, специальное приложение может предлагать учащимся вакансии, делить аудитории, высылая в разные части разный контент. Также в приложении может быть встроенный мессенджер и специальная кнопка для застенчивых студентов, которые стесняются задать вопрос на занятии. Уже есть приложения, которые справляются с некоторыми этими задачами, например, BeHere.
    \item
    \textit{Фестивали и конференции.} Посетителям высылают карту и расписание, справочную информацию о тех, кто выступает, и предлагают подойти к стендам. Крупнейшие площадки уже опробовали технологию: на SXSW с помощью iBeacon устраивали опросы, обсуждения среди участников и помогали разобраться, где что находится, а на кинофестивале «Трайбека» зрители могли сориентироваться, когда и где покажут нужный им фильм. Необычнее же всего технологию использовали на Каннском кинофестивале. Официальное приложение показывало, кто из гостей где находится, а также отображало ссылку на профиль каждого в LinkedIn — так было удобнее налаживать бизнес-контакты.
\end{itemize}

Благодаря большим коммерческим перспективам, направление indoor-на\-ви\-га\-ции становится всё более востребованным и уже привлекло внимание таких крупных игроков на рынке, как Google, Apple, Qualcomm, Broadcom, Sony и другие, и в это, без сомнения, перспективное направление уже инвестируются сотни миллионов долларов \cite{web:meetBeacons}.

Всемирно известная компания PayPal обозначила свой курс развития, как «инвестиции в мобильные технологии, а также возрождение розничной торговли с помощью, опять же, технологий» \cite{web:PayPal}.

Разрабатываемое ими устройство – PayPal Beacon – как раз реализованный шаг в данном направлении. Оно представляет собой портативный считыватель дебетовый и кредитных карт, по функциональности не отличающийся от тех, что зачастую установлены на кассах. С помощью него, кроме непосредственно оплаты, возможно и локационное взаимодействие: местоположение пользователя является связующим звеном, помогающим определить, что он находится в конкретном заведении, и мгновенно предложить авторизацию для совершения дальнейших покупок.