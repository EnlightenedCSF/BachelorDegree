\subsection{Реализация алгоритма, основанного на сравнении локационных отпечатков}

Локационные «отпечатки пальцев» – набор признаков, присвоенных некоторой заведомо известной точке пространства \cite{elbes2013precise}.  В рамках задачи будем считать, что этим набором признаком является вектор из эталонных показателей RSSI, принятых от каждого из маячков.

Смысл алгоритма тогда будет состоять в следующем: пользователь в некоторый момент времени определяет ряд значений RSSI. Этот набор сравнивается с остальными с целью нахождения ближайшего «отпечатка», значения которого отличается минимально в рамках принятой метрики. И так как положение каждого из «отпечатка» заранее известно, становится возможным определить положение пользователя.

Данную задачу можно рассматривать как задачу классификации. Так, каждый отпечаток $FP_i$ в точке пространства $i \quad (i=\overline{1,N})$ может быть представлен как $FP_i = \{ RSSI_1^{FP_i}, RSSI_2^{FP_i}, ..., RSSI_M^{FP_i} \}$, и соответствующая ему локация есть $L_i^{FP} = (X_i, Y_i)$.

Если представить положение пользователя как $T$, то ему сопоставим отпечаток $ S = \{ RSSI_1^T, RSSI_2^T, ..., RSSI_M^T \}$. После этого возможно найти дистанцию между $S$ и $FP_i$:
\begin{equation} \label{for:D}
    D^T_{FP_i} = \sqrt{ \sum_{j=1}^N ( RSSI_j^{FP_i} - RSSI_j^T )^2 }
\end{equation}

Будем считать $M$ количеством маячков, участвующих в рассмотрении. Теперь на основе расстояния можем вычислить вероятность того, что объект $Т$ находится около отпечатка $FP_i$ с измерениями $S$:
\begin{equation} \label{for:Prob}
    P(FP_i | S) = \frac{1}{ D^T_{FP_i} }
\end{equation}

Тогда правило определения локации можно сформулировать так: \textit{искомая точка – $i$ – если $P(FP_i | S) > P(FP_j | S)$, где $i,j = \overline{1,N}$. Ей соответствует локация $L_i^{FP}$.}

В реальности искомая позиция пользователя не является дискретной, и поэтому возникает необходимость интерполяции. Эта процедура осуществляется с помощью следующего преобразования, представленного ниже:
\[
    LT(x,y) = \sum_{i=1}^N P(FP_i | S) L_i^{FP}(x,y)
\]
где $(x,y)$ - координаты $FP$.

В работе \cite{elbes2013precise} предложено дополнение для данного подхода, называемое Budgeted Dynamic Exclusion (BDE, "бюджетное динамическое исключение"). Суть модификации сводится к следующему: сравниваются показания $j$ и отпечаток $i$. Если некоторая пара значений RSSI, соответствующих одному и тому же маяку, меньше установленной константы, именуемой бюджетом, то эти два значения выкидываются из вектора с остальными значениями RSSI. Это приводит к тому, что оцениваемое формулой \ref{for:D} Евклидово расстояние уменьшается, и, соответственно, степень доверия, вычисляемая с помощью формулы \ref{for:Prob}, увеличивается.

Главным препятствием в работе такой системы является дифракция, отражение и рассеяние сигнала, принимаемого от маяков \cite{liu2007survey}. Кроме того, возникают сложности, связанные с выходом из строя маячков. Ведь в таком случае необходимо либо переснимать показания сигнал-шума от оставшихся маячков заново, либо предусмотреть дополнительную логику в приложении и работе алгоритма в целом. Оба способа достаточно затратны по времени.

Хотя данный метод нахождения пользовательской локации был использован, например, на конференции GeekPicnic, проведенного в Москве в 2015г. \cite{web:habrGeekPicknic}, в итоговую версию библиотеки, разработанной в рамках данной работы, алгоритм не включен.