\section*{Введение}
\addcontentsline{toc}{section}{Введение}

В последнее время всё более актуальной становится проблема навигации внутри помещений, а также предоставления посетителям услуг, основанных на их местоположении и предпочтениях. Здания становятся всё более объёмными и нередко имеют довольно сложную структуру, ориентироваться в которой могут лишь те, кто постоянно посещает такие здания, а для неподготовленного человека ориентирование в таких местах превращается в пытку.

Благодаря indoor-навигации (навигации внутри помещений) появляются новые инструменты для маркетинга: проходя мимо магазина, человек может моментально узнать о проводимых в нем акциях, мероприятиях, товарах и предоставляемых услугах, благодаря всплывающему сообщению на экране своего телефона (так называемом “Geo-fencing”, причём предложенные ему предложения будут учитывать его интересы, так как можно учитывать информацию о его прошлых покупках), либо просто получить уведомление при приближении к определенному месту (второе направление indoor-навигации, называемое «Geo-aware»), а владельцы – получать статистическую информацию («тепловые карты» посетителей – своеобразный и очень мощный offline-аналог Google Analytics), основанные на перемещениях клиентов внутри торговых залов (таким образом понять, какие отделы и товары пользуются повышенным интересом, очень легко). Рынок подобной геоконтекстной рекламы уже измеряется миллиардами долларов, и с развитием систем indoor-навигации ожидается его стремительный рост.

Кроме того, решения, применяемые в indoor-навигации, помогают и в ориентировании вне зданий, на улице – там, где в условиях плотной застройки использование систем спутниковой навигации затруднено. Особенно эта проблема актуальна для Японии с высокой плотностью городской застройки.

Bluetooth-маяки, представляющие собой один из способов решения проблемы локации, также являются частью Интернета вещей, и способны изменить представление о самом Интернете вещей. Google представила концепцию "physical web", которая призвана объединить два мира: реальный и виртуальную Интернет-сеть. Смартфон, сканируя ближайшие к нему маячки, получает из их сигнала встроенные URL-ссылки, доступные пользователю. Таким образом, пользователь может мгновенно получить доступ к информации, определенной в ближайшем локационном контексте. Кроме того, это шаг навстречу унифицированному, бесшовному интерфейсу взаимодействия с цифровым миром.

Но современный этап развития технологии внутренней навигации представлен в основном технологиями для навигации роботов по маякам. Кроме того, существует множество готовых продуктов и платформ от производителей с мировым именем. В то же время сообщество программистов не располагает подходящими материалами, чтобы оценить существующие варианты алгоритмов трилатерации и выбрать лучший способ в рамках собственных приложений или свободно распространяемыми библиотеками, которые можно было бы легко подстроить в рамках решаемой задачи.

Именно это и подтолкнуло меня к решению задач, поставленных в рамках настоящей работы.