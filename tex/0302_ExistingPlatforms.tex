\subsection{Анализ существующих платформ}

В рамках рассматриваемой работы были проанализированы главные существующие платформы, способные без программирования строить iBeacon-приложения, а также выделены их главные особенности.

\begin{itemize}
    \item
    Indoors (http://indoo.rs/). Компания представляет целую платформу для навигации в помещениях. В перечень инструментов входят инструменты для моделирования пространства, расстановки маяков и их калибровки. С результами работы можно ознакомиться в демонстрационном приложении.
    \item
    LabWerk (http://labwerk.com). Основной продукт данной корпорации - решение для музеев mApp. Как следствие, существует функциональность для создания статей об объектах выставок и экспонатах, а также социальные элементы вроде создания опросов. Предусмотрена и карта заведения, отображающее текущее положение пользователя. Кроме того, система умеет собирать статистику о посещениях.
    \item
    ShopJoy (http://shopjoy.se). Авторы данной платформы предоставляют возможность тонко настраивать push-уведомления для всех клиентов на основе их персонализации. За основу берется информация, указанная пользователем при регистрации.
    \item
    LocalSocial (https://www.mylocalsocial.com). В целом у продукта можно выделить две сильные стороны. Во-первых, это система управления содержимым, через которую легко выделить географические зоны и зарегистрировать в них маячки. Во-вторых, это система лояльности пользователей: возможна настройка спец\-пред\-ло\-же\-ний, промо-акций, начисление очков лояльности и выдача промо-кодов.
\end{itemize}