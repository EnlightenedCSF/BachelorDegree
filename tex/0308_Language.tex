\subsection{Выбор языка и технологий программирования}

\textbf{Objective-C} -- объектно-ориентированный язык программирования. \\Язык во многом основывается на парадигмах, предложенных языком Small\-talk, и построенн на основе языка С. Из Smalltalk-черт можно выделить объектную модель: метод экземпляра не вызывается напрямую, для вызова объекту посылается сообщение. Аналогично, класс или объект могут определить, способны ли они вызвать указанный метод с указанными параметрами.

Из-за родства с языком C, компилятор понимает как Objective-C, так и C код.

Компилятор Objective-C входит в GCC и доступен на большинстве основных платформ. Язык используется в первую очередь для Mac OS X (Cocoa), GNUstep и iOS (Cocoa Touch).

\textbf{Swift} -- мультипарадигменный объектно-ориентированный язык программирования, созданный компанией Apple для разработчиков iOS и OS X. Swift работает с библиотеками Cocoa и Cocoa Touch и совместим с основной кодовой базой Apple, написанной на Objective-C. Swift задумывался как более безопасный язык в сравнении с Objective-C. Язык поддерживается в среде программирования Xcode; программы на нем компилируются при помощи Apple LLVM и используют рантайм Objective-C, что делает возможным использование обоих языков (а также чистого С и С++) в рамках одной программы.

Для реализации библиотеки был выбран язык Objective-C, так как язык хорошо поддерживается стандартной IDE “XCode”, отлично документирован и обладает широким сообществом программистов. Кроме того, в Objective-C лучшая по сравнению со Swift поддержка вставок кода на C, а именно на нем написаны алгоритмы трилатерации, рассматриваемые в данной работе.