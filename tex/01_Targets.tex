\section{Постановка задачи}

Цель работы:
\begin{enumerate}
    \item
    Проанализировать и сравнить различные методы трилатерации, используемые при решении задачии навигации внутри помещений:
    \begin{enumerate}
        \item
        метод, основанный на поиске области пересечения сфер;
        \item
        метод, основанный на поиске силового центра;
        \item
        продвинутый геометрический алгоритм;
        \item
        метод, основанный на фильтрации частиц;
        \item
        метод, основанный на сравнении отпечатков измерений.
    \end{enumerate}
    \item
    Разработать библиотеку, содержащую реализации наиболее перспективных и гибких алгоритмов из перечисленных выше, выбранных на основе предварительного анализа.
\end{enumerate}

К разрабатываемому проекту выдвинуты следующие требования:
\begin{enumerate}
    \item
    Поддержка различных методов трилатерации;
    \item
    Работа алгоритмов должна быть максимально оптимизирована как по скорости вычислений, так и по объему занимаемой памяти;
    \item
    Библиотека должна легко подключаться в проект при помощи менеджеров управления зависимостей CocoaPods или Carthage;
    \item
    Первоначальная настройка библиотеки в проекте должна содержать минимум кода и быть интуитивно понятной.
\end{enumerate}