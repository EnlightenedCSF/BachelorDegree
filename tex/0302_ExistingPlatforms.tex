\subsection{Анализ существующих платформ}

В рамках рассматриваемой работы были проанализированы главные существующие платформы, способные без программирования строить iBeacon-приложения, а также выделены их главные особенности.

\begin{enumerate}
    \item
    Indoors (http://indoo.rs/). Платформа для indoor-навигации. Поставляется с инструментами для моделирования пространства, расстановки маяков в нем, калибровки и т.д. Демонстрационное приложение позволяет увидеть результаты моделирования.
    \item
    LabWerk (http://labwerk.com). Основной продукт - решение для музеев mApp. Присутствует CMS для создания статей об объектах музея и настройки правил уведомления. Возможно создание опросов, привязанных к определенным регионам в музее. Поддерживается многоязычность. О посетителях собирается статистика. Приложение выводит статьи об объектах искусства при приближении к ним. Есть карта музея, на которой отображается текущее местоположение посетителя.
    \item
    ShopJoy (http://shopjoy.se). Доступна возможность таргетировать объявления (представимых в виде push-нотификаций) по возрасту, полу и интересам пользователя, указанным при регистрации в приложении.
    \item
    LocalSocial (https://www.mylocalsocial.com). При помощи CMS можно настроить географические зоны, промо-акции и спец-предложения для них, зарегистрировать маяки и привязать их к определенным регионам, настроить push-уведомления при входе в регионы. Также реализована система лояльности, позволяющая зарабатывать очки лояльности и выдавать промо-коды.
\end{enumerate}