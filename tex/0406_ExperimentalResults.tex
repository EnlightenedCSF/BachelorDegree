\subsection{Влияние помех различного происхождения на работу алгоритмов}

Предложенная архитектура библиотеки предполагает получение синергитеческого эффекта от использования различных трилатерационных методов для достижения наилучшей итоговой точности.

Кроме того, в работе маячков существуют некоторые особенности, которые необходимо учитывать. Из этого следует вывод, что именно комбинация алгоритмов дает более надежный результат, ведь некоторые алгоритмы могут показывать себя сильнее или слабее в некоторых сценариях.

К отрицательным сторонам используемых маячков можно отнести следующие особенности:
\begin{itemize}
    \item
    \textbf{Периодическое пропадание сигнала}. Раз в некоторый промежуток времени (вплоть до 5с) величина сигнала не может быть вычислена правильно, и будет отражено значение $-1$. Для нивелирования данного эффекта можно, например, принимать предыдущее значение, полученное от маячка, считая, что оно успело измениться незначительно.
    \item
    \textbf{Интерференция сигналов}. Вследствие интерференции сигналов при неправильной установке маячков данные могут отличаться от реальных достаточно значительно. Следует принимать во внимание как принципы оптимального расположения маячков, описанные в текущей работе, так и плотность расположения маячков в помещениях. Даже для достаточно больших комнат или залов не следует монтировать маячки вдоль их периметра чаще, чем одно устройство на 4м периметра.
    \item
    \textbf{Помехи и препятствия}. К отображению неверных данных может и привести, конечно, наличие препятствия между устройством пользователя и маяком. В этом можно легко убедиться, стоя напротив маяка и, снимая показатели сигнала, повернуться спиной к источнику. В данном случае в качестве рекомендации, опять же, можно привести советы по оптимальному расположению BLE-устройств.
\end{itemize}

В проведении ряда экспериментов отмечено, что на адаптивный геометрический алгоритм общий шум показателей сигналов влияет в меньшей степени: с геометрической точки зрения, большие показатели - большие радиусы окружностей, и центроид точек пересечения, как результат, смещается незначительно.

Можно утверждать, что используемые методы надежны: каждый способен сформировать конечный результат, даже если полученные показания одновременно малы или велики.


