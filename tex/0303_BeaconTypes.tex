\subsection{Выбор типа маячков}

На рынке маячков уже десятки компаний предлагают собственные продукты, обладающие различными преимуществами и недостатками. Основными критериями для сравнения маячков являются:

\begin{itemize}
    \item
    \textbf{Базовая технология}, то есть непосредственно канал передачи. В основном это BLE, но возможны и другие варианты, в том числе и сочетание нескольких.
    \item
    \textbf{Совместимость}. Apple выдвинул собственный стандарт iBeacon, который не является единым и унифицированным, поэтому производитель вправе создать собственный стандарт передачи.
    \item
    \textbf{Параметры маячка}: мощность, частота обновления и другие.
    \item
    \textbf{Настройка и перенастройка}: прошивка маячков, их параметры; как производитель предполагает изменение этих параметров разработчиками.
    \item
    \textbf{Платформа и сервисы}. Сюда можно отнести такие характеристики, как SDK, единое управление, CMS и другие.
    \item
    \textbf{Безопасность}. Наличие шифрования, защиты от DoS-атак и прочие аспекты.
\end{itemize}

Среди основных производителей можно выделить следующие:

\textbf{Estimote}. Используют стандарт iBeacon через канал BLE. Особенностью является встроенный сенсор температуры и акселерометр.

\textbf{Gimbal}. Помимо iBeacon, используют собственный стандарт передачи. Обязательным требованием является регистрация приложений и маячков на базе Gimbal SDK. Платформа для предприятий удовлетворяет спецификации REST.

\textbf{StickNFind}. Использует стандарт iBeacon, предлагает варианты его расширения. Опционально шифрует пакеты. Возможность получения температуры и состояния батареи. Звуковая и световая сигнализация.