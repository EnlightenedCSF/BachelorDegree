\subsection{Анализ примеров успешного внедрения}

Международная платежная система MasterCard, Мультимедиа Арт Музей Москва, агентства Digitalizm и Insight ONE объявляют о запуске интерактивного гида <<MasterCard Бесценные города – Твой МАММ>> - первого в России интерактивного гида по музею, разработанного с применением технологии iBeacon. Таким образом проект решает современными инновационными средствами образовательные и маркетинговые задачи в пространстве музея \cite{web:firstMuseum}.

Крупная сеть аптек Rite Aid установила маячки в более чем 4500 торговых точках, воспользовавшись услугами компании inMarket. inMarket знаменит услугами по изготовлению и установке Bluetooth-маячков собственного, защищенного от атак хакеров, формата \cite{web:RiteAid}.

В другом материале InMarket описывает кейс американского производителя продуктов питания Hillshire Brands, который хотел повысить продажи определенного сорта выпускаемых им сосисок. Разработав и внедрив приложение, работающее на основе iBeacon, компания получила увеличение узнаваемости марки сосисок на 36\% и рост общего объема продаж в 10 американских магазинах, где были установлены маячки iBeacon \cite{web:Hillshire}.

Zatarain - первая в мире компания по производству фасованных потребительских товаров, которая внедрила технологию iBeacon. В своих магазинах компания в определенных контрольных точках установила маячки iBeacon. Оказываясь рядом с такими точками, посетители получали push-уведомления, которые мотивировали покупателей искать продукты Zatarain в магазине. Люди сканировали продукты бренда и получали за это призовые очки. Покупатели взаимодействовали с брендированнной мобильной страницей, держа в руках сам продукт от Zatarain. В итоге, люди, которые получали iBeacon-уведомления от компании, в 5 раз чаще использовали ее приложение, что соответственно привело к большому вовлечению потребителей в магазине. Оказалось, что посетители магазина, которые взаимодействовали с продуктом на контрольных точках, в 130 раз охотнее покупали его, чем остальные. Для сравнения -  обычный мобильный баннер увеличивает намерение делать покупки в среднем на 3\% \cite{web:Zatarin}.

Известный продавец обуви Timberland тестировал технологию iBeacon от Swirl в двух своих магазинах в Нью-Йорке и Бостоне. Результаты получились следующими: пользователи мобильного приложения просматривали 72\% рекламных предложений от общего количества приходящих на их смартфоны, а 35\% из них купили рекламируемый товар. Около 750 покупателей получили 20\% скидку от Timberland \cite{web:Timberland}.

“Virgin Atlantic” и аэропорт «Хитроу» тестируют сервис информирования авиапассажиров на основе маяков \cite{web:Heathrow}. В Московском аэропорту «Шереметьево» данная система уже частично в ходу, инженерам потребуется около 7500 устройств, чтобы покрыть площадь в 500 тысяч квадратных метров. Следом за «Шереметьево» к использованию iBeacon-навигации должны подключиться аэропорты «Домодедово» и «Внуково» \cite{web:Aeroports}.

Свидетельством этого является следующий факт: один из крупнейших мировых магазинов "Macy's"  планируют установить в залах своих торговых центров около 4000 таких устройств \cite{web:Macy}. 

В прошлом году на Российском Интернет Форуме <<РИФ+КИБ>> в здании было установлено 200 маячков, и всем посетителям была наглядно продемонстрирована система навигации внутри помещения \cite{web:RifKib}.

Выводы о внедрении:

Компания - производитель маячков Swirl провела исследование, результаты которого показали, что пользователи мобильных приложений, в первую очередь (80\%), желают получать push-уведомления о скидках. По данным того же исследования люди специально будут отключать возможность получения push-уведомлений, если в них не будет полезной информации (то есть тех же акций и скидок) или сообщения будут не релеванты их интересам или месторасположению.

Подтверждают эти выводы и исследования ювелирного магазина Alex\&Ani, внедрившего в свою работу iBeacon технологии. Вот его результаты:
\begin{itemize}
    \item
    93\% опрашиваемых покупателей согласилось с тем, что мобильные рекламные предложения, которые они получают в магазине увеличивают вероятность их покупки.
    \item
    90\% утверждают, что мобильные предложения заставляют их чаще посещать магазин.
    \item
    88\% сказали, что опыт пребывания в магазине становится более интуитивным и легким.
\end{itemize}

Cогласно исследованию Google, 84\% посетителей пользуются мобильными устройствами, находясь в магазинах, причём 50\% проводят в них не менее 15 минут \cite{web:GoogleShopExpirience}.

По данным аналитики, сегодня во время визита в магазин в 2 раза больше покупателей обращается за информацией о товарах и скидках к мобильному приложению, нежели к продавцу-консультанту. Мобильные решения сегодня полностью меняют картину маркетинговой активности потребителей \cite{web:CasesZero}.